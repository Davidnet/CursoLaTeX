\documentclass[11pt,a4paper]{article}
\usepackage{lmodern}
\usepackage[T1]{fontenc}
\usepackage[utf8]{inputenc}
\usepackage[spanish]{babel}
\usepackage{hyperref}
\usepackage{listings}
\usepackage{epigraph}

%%%
\usepackage{color}

\definecolor{mygreen}{rgb}{0,0.6,0}
\definecolor{mygray}{rgb}{0.5,0.5,0.5}
\definecolor{mymauve}{rgb}{0.58,0,0.82}

\lstset{ %
	backgroundcolor=\color{white},   % choose the background color; you must add \usepackage{color} or \usepackage{xcolor}
	basicstyle=\footnotesize,        % the size of the fonts that are used for the code
	breakatwhitespace=false,         % sets if automatic breaks should only happen at whitespace
	breaklines=true,                 % sets automatic line breaking
	captionpos=b,                    % sets the caption-position to bottom
	commentstyle=\color{mygreen},    % comment style
	deletekeywords={...},            % if you want to delete keywords from the given language
	escapeinside={\%*}{*)},          % if you want to add LaTeX within your code
	extendedchars=true,              % lets you use non-ASCII characters; for 8-bits encodings only, does not work with UTF-8
	frame=single,                    % adds a frame around the code
	keepspaces=true,                 % keeps spaces in text, useful for keeping indentation of code (possibly needs columns=flexible)
	keywordstyle=\color{blue},       % keyword style
	language=TeX,                 % the language of the code
	morekeywords={*,\begin, \documentclass},            % if you want to add more keywords to the set
	numbers=left,                    % where to put the line-numbers; possible values are (none, left, right)
	numbersep=5pt,                   % how far the line-numbers are from the code
	numberstyle=\tiny\color{mygray}, % the style that is used for the line-numbers
	rulecolor=\color{black},         % if not set, the frame-color may be changed on line-breaks within not-black text (e.g. comments (green here))
	showspaces=false,                % show spaces everywhere adding particular underscores; it overrides 'showstringspaces'
	showstringspaces=false,          % underline spaces within strings only
	showtabs=false,                  % show tabs within strings adding particular underscores
	stepnumber=2,                    % the step between two line-numbers. If it's 1, each line will be numbered
	stringstyle=\color{mymauve},     % string literal style
	tabsize=2,                       % sets default tabsize to 2 spaces
	title=\lstname                   % show the filename of files included with \lstinputlisting; also try caption instead of title
}
%%%


\author{David Cardozo}
\title{Guía de Instalación de Latex}
\begin{document}
\maketitle
\epigraph{¿Sabe usted? Todos nos hicimos matemáticos por la misma razón: éramos perezos.}{Max Rosenlicht}

Este es un documento que da las pasos básicos para obtener una ambiente informático correcto para el uso de \LaTeXe.

\textbf{Historia}

 \LaTeX{} es un software de preparación de documentos muy utilizados en el ámbito académico y tipografía en el mundo, la creación de \LaTeX{} fue sustanciada con la necesidad de ser un lenguaje de alto nivel y macros sobre el sistema \TeX{}. \LaTeX{}  es un ambiente de escritura fundamentado en el estilo WYWIWYM que significa \emph{What you see is what you meant}, cuyo énfasis es dejar que el autor solo se preocupe por la creación de texto, y no por la organización o estética del documento (\emph{e.g.} Word), \LaTeX{} requiere de tener configurado un buen ambiente de configuración pues es necesario para la instalación de paquetes adicionales que permiten expandir las características de \LaTeX{}. 
 
 \textbf{Entornos de desarrollo integrado}
 
\LaTeXe{}   puede ser considerado como un lenguaje de programación de alto nivel como lo son Java, Python y Visual Basic. Como lenguaje de programación, es posible crear programas cortos utilizando programas rudimentarios como son el Bloc de Notas en Windows o el Note App de Mac, pero a medida que se realizan proyectos de alta complejidad, como los son libros, artículos y presentaciones, es altamente recomendable el uso de un \emph{entorno de desarrollo } enfatizado en \LaTeXe{}. Actualmente existen diversos entornos en la web, muchos responden a diferentes necesidades, existen entornos en la nube (\href{https://www.sharelatex.com}{ShareLaTeX}) que tienen la desventaja de no tener suficientes paquetes de personalización; \emph{'Add-ins'} que son compiladores agregados a entornos de desarrollo de otros lenguajes, por ejemplo el proyecto \href{texlipse.sourceforge.net}{\textbf{TeXlipse}} cuyo propósito es de dotar un compilador en \TeX{} para Eclipse; y por ultimo, los entornos especializados, que para esta guía serán necesarios y se recomendara el proyecto de software libre \href{http://texstudio.sourceforge.net/}{\textbf{\TeX Studio}}, que permiten la instalación de paquetes avanzados y la personalización adaptada a usuarios.

\textbf{Pasos para la instalación de un ambiente en Windows}

\begin{enumerate}
	\item Visitar la pagina del proyecto \href{http://miktex.org/download}{ \textbf{Mik\TeX{}}} y bajar e instalar el ejecutable de ultima versión.
	\item Seguir los pasos instructivos en pantalla.
	\item Al terminar la instalación, reiniciar el equipo.
	\item  Realizar la instalación del ambiente \href{http://texstudio.sourceforge.net/}{\textbf{\TeX Studio}}, buscar la ultima versión.
	\item Al terminar la instalación reiniciar el equipo.
\end{enumerate}

\textbf{Creación del documento ¡Hola Mundo!}

Al terminar los pasos anteriores, el equipo tendrá en este momento una instalación básica de \LaTeXe. Una practica común para la prueba de cualquier instalación de un lenguaje de programación, es la creación de un programa que le presente al usuario el texto \emph{Hello World!}. Para ello, daremos doble clic sobre el nuevo programa \textbf{\TeX Studio} y sobre la ventana que se abre, iremos a ``\emph{File $ \rightarrow $ New}'' y esto crea un nuevo documento, sobre esta nueva planilla introduciremos el siguiente código:


\begin{lstlisting}
\documentclass[11pt]{article} 

\begin{document}
Hello World! \\ comentario utilizando  % %* \LaTeXe  *)
\end{document}
\end{lstlisting}

Al terminar, presionar la tecla F1, o ir a \emph{Tools $\rightarrow$ Build \& View }. 

Si el ambiente de desarrollo esta bien configurado, se  producirá un documento en PDF, que tendrá \emph{Hello World!} como texto.

Felicitaciones, usted a producido su primer texto en \LaTeXe{}, hasta aquí la primera lección. 






\end{document}